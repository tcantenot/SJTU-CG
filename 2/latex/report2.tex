\documentclass[a4paper,10pt]{article}

\usepackage[utf8]{inputenc}
\usepackage{mathtools}
\usepackage{amsfonts}
\usepackage{amsmath}

\title{\textbf{Computer Graphics} \\ Assignment 2}
\author{Thierry CANTENOT \\ J114030901}
\date{23/03/14}

\pdfinfo{%
  /Title    (Assignment 2)
  /Author   (Thierry CANTENOT)
  /Creator  ()
  /Producer ()
  /Subject  (Computer Graphics)
  /Keywords ()
}

\begin{document}
\maketitle

\section{Rotations}
\bigskip

\subsection{Problem statement}
\bigskip

\begin{enumerate}
\item Prove that two successive 2D rotations are additive:\\ $R(\theta_1) . R(\theta_2) = R(\theta_1+\theta_2)$.

\bigskip
\item Consider a line from the origin of a right-handed coordinate system to the point P(x,y,z). Find the transformation matrices needed to rotate the line into the positive z axis in two different ways, and show by algebraic manipulation that, in each case, the point P does go to the z axis. For each method, calculate the sines and cosines of the angles of rotation.
a. Rotate about the y axis into the (y,z) plane, then rotate about the x axis into the z axis.
b. Rotate about the z axis into the (x,z) plane, then rotate about the y axis into the z axis.

\end{enumerate}

\pagebreak
\subsection{Answer}
\bigskip

\subsection{Q1}
\bigskip
In 2D, a rotation of angle $\theta$ has the following matrix form:

\begin{equation}
R(\theta) =
\begin{bmatrix}
	cos(\theta) & -sin(\theta)\\
	sin(\theta) & cos(\theta)
\end{bmatrix}
\end{equation}
\noindent
Thus,
\begin{equation}
\left.\begin{aligned}
R(\theta_1).R(\theta_2)&=
\begin{bmatrix}
	cos(\theta_1) & -sin(\theta_1)\\
	sin(\theta_1) & cos(\theta_1)
\end{bmatrix}
\begin{bmatrix}
	cos(\theta_2) & -sin(\theta_2)\\
	sin(\theta_2) & cos(\theta_2)
\end{bmatrix}&\\
&= \begin{bmatrix}
	cos(\theta_1)cos(\theta_2) - sin(\theta_1)sin(\theta_2) & -cos(\theta_1)sin(\theta_2) - sin(\theta_1)cos(\theta_2)\\
	sin(\theta_1)cos(\theta_2) + cos(\theta_1)sin(\theta_2) & -sin(\theta_1)sin(\theta_2) + cos(\theta_1)cos(\theta_2)
\end{bmatrix}&
\end{aligned}\right.
\end{equation}

\bigskip \noindent
By considering the following trigonometric equalities:

\begin{equation}
\left.\begin{aligned}
&cos(\alpha)cos(\beta) - sin(\alpha)sin(\beta) = cos(\alpha + \beta)&\\
&sin(\alpha)cos(\beta) + cos(\alpha)sin(\beta) = sin(\alpha + \beta)&
\end{aligned}\right.
\end{equation}

\bigskip \noindent
And then by replacing, we have

\begin{equation}
\left.\begin{aligned}
R(\theta_1).R(\theta_2)&=
\begin{bmatrix}
	cos(\theta_1 + \theta_2) & -sin(\theta_1 + \theta_2)\\
	sin(\theta_1 + \theta_2) & cos(\theta_1 + \theta_2)
\end{bmatrix}&\\
&= R(\theta_1 + \theta_2) \qquad \text{Q.E.D}&\\
\end{aligned}\right.
\end{equation}

\pagebreak

\subsubsection{Q2}
\bigskip

\noindent
Let $O$ be the origin of our right-handed coordinate system. \\
$OP$ denotes the vector from $O(0, 0, 0)$ to $P(x, y, z)$ and $|OP|$ its length.


\bigskip \noindent
\textbf{a})

\bigskip \noindent
\textbf{1/ Rotate $OP$ about the $Y$ axis onto the $(YZ)$ plane:}\\\\
Let $P_{xz}$ the projection of the point $P$ on the plane $XZ$ and $\theta$ be the angle between $P_{xz}$ and the $X$ axis.

\bigskip \noindent
The rotation we want to perform correspond to the angle $-(90 - \theta) = \theta - 90$. \\\\
We know that the 4x4 rotation about an angle $\alpha$ around the $Y$ axis has the following matrix form:

\begin{equation}
R_y(\alpha) =
\begin{bmatrix}
	cos(\alpha)  & 0 & sin(\alpha) & 0 \\
	0 			 & 1 & 0		   & 0 \\
	-sin(\alpha) & 0 & cos(\alpha) & 0 \\
	0 			 & 0 & 0 		   & 1
\end{bmatrix}
\end{equation}
\noindent
In our case, $\alpha = 90 - \theta$ and
\begin{equation}
\left.\begin{aligned}
	&cos(\alpha) = \frac{z}{D}& \\
	&sin(\alpha) = -\frac{x}{D}&
\end{aligned}\right.
\end{equation}
where $D = \sqrt{x^2 + z^2}$.\\


\noindent
Let $P'$ be the rotated point $P$ belonging to the $(YZ)$ plane.

\begin{equation}
\left.\begin{aligned}
P' = R_y(\alpha)P 
&=
\begin{bmatrix}
	cos(\alpha)  & 0 & sin(\alpha) & 0 \\
	0 			 & 1 & 0		   & 0 \\
	-sin(\alpha) & 0 & cos(\alpha) & 0 \\
	0 			 & 0 & 0 		   & 1
\end{bmatrix}
\begin{bmatrix}
x \\ y \\ z \\ 1
\end{bmatrix}&\\
&=
\begin{bmatrix}
	\frac{z}{D}  & 0 & -\frac{x}{D} & 0 \\
	0 			 & 1 & 0		    & 0 \\
	\frac{x}{D}  & 0 & \frac{z}{D}  & 0 \\
	0 			 & 0 & 0 		    & 1
\end{bmatrix}
\begin{bmatrix}
x \\ y \\ z \\ 1
\end{bmatrix}&\\
&=
\begin{bmatrix}
\frac{xz}{D} + \frac{-zx}{D} \\
 y \\ 
 \frac{x^2}{d} + \frac{z^2}{D} \\ 
 1
\end{bmatrix}&\\
&=
\begin{bmatrix}
0 & y & \sqrt{x^2 + z^2} & 1
\end{bmatrix}^{T}&\\
&=
\begin{bmatrix}
0 & y & D & 1
\end{bmatrix}^{T}&
\end{aligned}\right.
\end{equation}


\pagebreak
\noindent
\textbf{2/ Rotate $OP'$ about the $X$ axis onto the $Z$ axis:}


\bigskip \noindent
Let $\phi$ be the angle of rotation around the $X$.

\bigskip \noindent
We know that the 4x4 rotation about an angle $\alpha$ around the $X$ axis has the following matrix form:

\begin{equation}
R_x(\alpha) =
\begin{bmatrix}
	1  & 0 			& 0 		   & 0 \\
	0 & cos(\alpha) & -sin(\alpha) & 0 \\
	0 & sin(\alpha) & cos(\alpha)  & 0 \\
	0 & 0 			& 0 		   & 1
\end{bmatrix}
\end{equation}

\noindent
In our case $\alpha = \phi$, and since a rotation conserve the length (of $OP$)
\begin{equation}
\left.\begin{aligned}
&cos(\alpha) = \frac{D}{|OP|} = \frac{\sqrt{x^2 + z^2}}{\sqrt{x^2 + y^2 + z^2}}&\\
&sin(\alpha) = \frac{y}{|OP|} = \frac{y}{\sqrt{x^2 + y^2 + z^2}}&\\
\end{aligned}\right.
\end{equation}

\bigskip \noindent
Let $P''$ be the rotated point $P'$ belonging to the $Z$ axis.

\begin{equation}
\left.\begin{aligned}
P'' = R_x(\alpha)P'
&=
\begin{bmatrix}
	1 & 0 			& 0 		   & 0 \\
	0 & cos(\alpha) & -sin(\alpha) & 0 \\
	0 & sin(\alpha) & cos(\alpha)  & 0 \\
	0 & 0 			& 0 		   & 1
\end{bmatrix}
\begin{bmatrix}
0 \\ y \\ D \\ 1
\end{bmatrix}&\\
&=
\begin{bmatrix}
	1 & 0 			   & 0 		         & 0 \\
	0 & \frac{D}{|OP|} & -\frac{y}{|OP|} & 0 \\
	0 & \frac{y}{|OP|} & \frac{D}{|OP|}  & 0 \\
	0 & 0 			   & 0 		         & 1
\end{bmatrix}
\begin{bmatrix}
0 \\ y \\ D \\ 1
\end{bmatrix}&\\
&=
\begin{bmatrix}
0 & \frac{yD}{|OP|} + \frac{-Dy}{|OP|} & \frac{y^2 + D^2}{|OP|} & 1
\end{bmatrix}^{T}&\\
&=
\begin{bmatrix}
0 & 0 & \frac{y^2 + (\sqrt{x^2 + z^2})^2}{\sqrt{x^2 + y^2 + z^2}} & 1
\end{bmatrix}^{T}&\\
&=
\begin{bmatrix}
0 & 0 & |OP| & 1
\end{bmatrix}^{T}&\\
\end{aligned}\right.
\end{equation}

\bigskip \bigskip \noindent
\textbf{3/ Full transformation:}

\bigskip \noindent
The full transformation consists in the concatenation of the two rotations we applied:
\begin{equation}
\left.\begin{aligned}
P'' &= R_x(\phi)R_y(90 - \theta)P&\\
&=
\begin{bmatrix}
0 & 0 & |OP| & 1
\end{bmatrix}^{T}&\\
&=
\begin{bmatrix}
0 & 0 & \sqrt{x^2 + y^2 + z^2} & 1
\end{bmatrix}^{T}&\\
\end{aligned}\right.
\end{equation}




\pagebreak \noindent
\textbf{b})

\bigskip \noindent
\textbf{1/ Rotate $OP$ about the $Z$ axis onto the $(XZ)$ plane:}\\\\
Let $P_{xy}$ the projection of the point $P$ on the plane $XY$ and $\theta$ be the angle between $P_{xy}$ and the $Y$ axis.

\bigskip \noindent
The rotation we want to perform correspond to the angle $-(90 - \theta) = \theta - 90$. \\\\
We know that the 4x4 rotation about an angle $\alpha$ around the $Y$ axis has the following matrix form:

\begin{equation}
R_z(\alpha) =
\begin{bmatrix}
	cos(\alpha) & -sin(\alpha) & 0 & 0 \\
	sin(\alpha) & cos(\alpha)  & 0 & 0 \\
	0 		    & 0 		   & 1 & 0 \\
	0 			& 0 		   & 0 & 1
\end{bmatrix}
\end{equation}
\noindent
In our case, $\alpha = \theta - 90$ and
\begin{equation}
\left.\begin{aligned}
	&cos(\alpha) = \frac{x}{D}& \\
	&sin(\alpha) = -\frac{y}{D}&
\end{aligned}\right.
\end{equation}
where $D = \sqrt{x^2 + y^2}$.\\


\noindent
Let $P'$ be the rotated point $P$ belonging to the $(XY)$ plane.

\begin{equation}
\left.\begin{aligned}
P' = R_z(\alpha)P 
&=
\begin{bmatrix}
	cos(\alpha) & -sin(\alpha) & 0 & 0 \\
	sin(\alpha) & cos(\alpha)  & 0 & 0 \\
	0 		    & 0 		   & 1 & 0 \\
	0 			& 0 		   & 0 & 1
\end{bmatrix}
\begin{bmatrix}
x \\ y \\ z \\ 1
\end{bmatrix}&\\
&=
\begin{bmatrix}
	\frac{x}{D}  & \frac{y}{D} & 0 & 0 \\
	-\frac{y}{D} & \frac{x}{D}  & 0 & 0 \\
	0 		     & 0 		    & 1 & 0 \\
	0 			 & 0 		    & 0 & 1
\end{bmatrix}
\begin{bmatrix}
x \\ y \\ z \\ 1
\end{bmatrix}&\\
&=
\begin{bmatrix}
\frac{x^2}{D} + \frac{y^2}{D} \\
 \frac{-xy}{d} + \frac{yx}{D} \\ 
 z \\
 1
\end{bmatrix}&\\
&=
\begin{bmatrix}
\sqrt{x^2 + y^2} & 0 & z & 1
\end{bmatrix}^{T}&\\
&=
\begin{bmatrix}
D & 0 & z & 1
\end{bmatrix}^{T}&
\end{aligned}\right.
\end{equation}


\pagebreak
\noindent
\textbf{2/ Rotate $OP'$ about the $Y$ axis onto the $Z$ axis:}


\bigskip \noindent
Let $\phi$ be the angle of rotation around the $Y$.

\bigskip \noindent
We know that the 4x4 rotation about an angle $\alpha$ around the $Y$ axis has the following matrix form:

\begin{equation}
R_y(\alpha) =
\begin{bmatrix}
	cos(\alpha)  & 0 & sin(\alpha) & 0 \\
	0 			 & 1 & 0		   & 0 \\
	-sin(\alpha) & 0 & cos(\alpha) & 0 \\
	0 			 & 0 & 0 		   & 1
\end{bmatrix}
\end{equation}

\noindent
In our case $\alpha = 90 - \phi$, and since a rotation conserve the length (of $OP$)
\begin{equation}
\left.\begin{aligned}
&cos(\alpha) = \frac{z}{|OP|} = \frac{z}{\sqrt{x^2 + y^2 + z^2}}&\\
&sin(\alpha) = -\frac{D}{|OP|} = -\frac{\sqrt{x^2 + y^2}}{\sqrt{x^2 + y^2 + z^2}}&\\
\end{aligned}\right.
\end{equation}

\bigskip \noindent
Let $P''$ be the rotated point $P'$ belonging to the $Z$ axis.

\begin{equation}
\left.\begin{aligned}
P'' = R_y(\alpha)P'
&=
\begin{bmatrix}
	cos(\alpha)  & 0 & sin(\alpha) & 0 \\
	0 			 & 1 & 0		   & 0 \\
	-sin(\alpha) & 0 & cos(\alpha) & 0 \\
	0 			 & 0 & 0 		   & 1
\end{bmatrix}
\begin{bmatrix}
D \\ 0 \\ z \\ 1
\end{bmatrix}&\\
&=
\begin{bmatrix}
	\frac{z}{|OP|} & 0 & -\frac{D}{|OP|} & 0 \\
	0 			   & 1 & 0		   		 & 0 \\
	\frac{D}{|OP|} & 0 & \frac{z}{|OP|} & 0 \\
	0 			   & 0 & 0 		     & 1
\end{bmatrix}	
\begin{bmatrix}
D \\ 0 \\ z \\ 1
\end{bmatrix}&\\
&=
\begin{bmatrix}
\frac{Dz}{|OP|} + \frac{-zD}{|OP|} & 0 & \frac{z^2 + D^2}{|OP|} & 1
\end{bmatrix}^{T}&\\
&=
\begin{bmatrix}
0 & 0 & \frac{z^2 + (\sqrt{x^2 + y^2})^2}{\sqrt{x^2 + y^2 + z^2}} & 1
\end{bmatrix}^{T}&\\
&=
\begin{bmatrix}
0 & 0 & |OP| & 1
\end{bmatrix}^{T}&\\
\end{aligned}\right.
\end{equation}

\bigskip \bigskip \noindent
\textbf{3/ Full transformation:}

\bigskip \noindent
The full transformation consists in the concatenation of the two rotations we applied:
\begin{equation}
\left.\begin{aligned}
P'' &= R_y(90 - \phi)R_z(90 - \theta)P&\\
&=
\begin{bmatrix}
0 & 0 & |OP| & 1
\end{bmatrix}^{T}&\\
&=
\begin{bmatrix}
0 & 0 & \sqrt{x^2 + y^2 + z^2} & 1
\end{bmatrix}^{T}&\\
\end{aligned}\right.
\end{equation}

\end{document}
